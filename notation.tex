%% 日本語テキストの中で単位付きの量を書くとき,数と単位の間にいれるスペース
%%  日本語文のなかで「~」使うと,それは「\hskip\xkanjiskip」より広いので,
%%  数字と単位がだいぶ離れているように見えてしまうという問題が発生する.
\newcommand{\xu}{\nolinebreak\hskip.9\xkanjiskip\relax}

\def\-{--}%                %% 固有名詞を連結するときのハイフンを en dash にする
\def\hseq{\quad}%           %% 横方向に数式を並べるときのスペース
\newcommand{\GeneralBracket}[4]{%
  \ifx#1*\left#2#4\right#3%
  \else\ifx#1.\mathopen#2#4\mathclose#3%
  \else\mathopen#1#2#4\mathclose#1#3\fi\fi}

\makeatletter
\newcommand{\GeneralPair}[6]{%
  \ifx#1*\GeneralPair@v{#2}{#3}{#4}{#5}{#6}%
  \else\ifx#1.\mathopen#2#5\!\!\;\mathbin#3\!\!\;#6\mathclose#4%
  \else\mathopen#1#2#5\mathbin#1#3#6\mathclose#1#4\fi\fi}
\newcommand{\GeneralPair@v}[5]{\mathchoice%
  {\GeneralPair@vs{\displaystyle}{#1}{#2}{#3}{#4}{#5}}%
  {\GeneralPair@vs{\textstyle}{#1}{#2}{#3}{#4}{#5}}%
  {\GeneralPair@vs{\scriptstyle}{#1}{#2}{#3}{#4}{#5}}%
  {\GeneralPair@vs{\scriptscriptstyle}{#1}{#2}{#3}{#4}{#5}}}
\newcommand{\GeneralPair@vs}[6]{%
  \setbox0=\hbox{$#1#5$}\setbox2=\null\ht2=\ht0\dp2=\dp0%
  \setbox1=\hbox{$#1#6$}\setbox3=\null\ht3=\ht1\dp3=\dp1%
  \left#2\box0\left#3\box2\box3\right.\kern-\nulldelimiterspace\box1\right#4}

\newcommand{\pvPair}[3][*]{\GeneralPair{#1}(|){#2}{#3}}
\newcommand{\pVPair}[3][*]{\GeneralPair{#1}(\|){#2}{#3}}
\newcommand{\bvPair}[3][*]{\GeneralPair{#1}[|]{\!\:#2}{#3\!\:}}
\newcommand{\BvPair}[3][*]{\GeneralPair{#1}\{|\}{#2}{#3}}
\newcommand{\pvpair}[3][.]{\GeneralPair{#1}(|){#2}{#3}}
\newcommand{\pVpair}[3][.]{\GeneralPair{#1}(\|){#2}{#3}}
\newcommand{\bvpair}[3][.]{\GeneralPair{#1}[|]{\!\:#2}{#3\!\:}}
\newcommand{\Bvpair}[3][.]{\GeneralPair{#1}\{|\}{#2}{#3}}

\newcommand{\fvPair}[3][*]{\!\pvPair[#1]{#2}{#3}}% ( | ) 括弧が可変
\newcommand{\fvpair}[3][.]{\pvpair[#1]{#2}{#3}}% ( | ) 括弧が固定

%「FunctionName (左側|右側)」と出したいとき
% \fnvPair{FunctionName}({左側}|{右側}) % ( | ) 括弧が可変(直前に関数名がくる)
% \fnvpair{FunctionName}({左側}|{右側}) % ( | ) 括弧が固定(直前に関数名がくる)
% \fnvPair[\big]{FunctionName}({左側}|{右側}) % とすると( | ) を \big にする
% ↓引数のなかに(|)のいずれが入っているときは({}|{})のように囲むと安全.
\newcommand{\fnvPair}[2][*]{\@ifnextchar({\fnvpPair@{#2}{#1}}{\@ifnextchar[{\fnvbPair@{#2}{#1}}{\GenerateWarning{\string\fnvPair failed}#2}}}
\newcommand{\fnvpair}[2][.]{\@ifnextchar({\fnvppair@{#2}{#1}}{\@ifnextchar[{\fnvbpair@{#2}{#1}}{\GenerateWarning{\string\fnvpair failed}#2}}}
\def\fnvpPair@#1#2(#3|#4){#1\!\pvPair[#2]{#3}{#4}}% ( | ) 括弧が可変(直前に関数名がくる)
\def\fnvppair@#1#2(#3|#4){#1\pvpair[#2]{#3}{#4}}% ( | ) 括弧が可変(直前に関数名がくる)
\def\fnvbPair@#1#2[#3|#4]{#1\!\bvPair[#2]{#3}{#4}}% [ | ] 括弧が可変(直前に関数名がくる)
\def\fnvbpair@#1#2[#3|#4]{#1\bvpair[#2]{#3}{#4}}% [ | ] 括弧が可変(直前に関数名がくる)


% % \ProbW({左側}|{右側}) は \fnvPair P({左側}|{右側}) のかわり
% % \Probw({左側}|{右側}) は \fnvpair P({左側}|{右側}) のかわり
% % \probW({左側}|{右側}) は \fnvPair p({左側}|{右側}) のかわり
% % \probw({左側}|{右側}) は \fnvpair p({左側}|{右側}) のかわり

%「P(左側|右側)」とか「p(左側|右側)」と出したいとき
% \ProbW({左側}|{右側}) % ( | ) 括弧が可変(直前に関数名がくる)
% \Probw({左側}|{右側}) % ( | ) 括弧が固定(直前に関数名がくる)
% \ProbW[\big]{FunctionName}({左側}|{右側}) % とすると( | ) を \big にする
% \probW, \probw では p は小文字になる
% ↓引数のなかに(|)のいずれが入っているときは({}|{})のように囲むと安全.
% w は when/where のイメージ
\newcommand{\ProbW}[1][*]{\ProbW@{#1}}
\newcommand{\Probw}[1][.]{\Probw@{#1}}
\newcommand{\probW}[1][*]{\probW@{#1}}
\newcommand{\probw}[1][.]{\probw@{#1}}
\def\ProbW@#1(#2|#3){P\fvPair[#1]{#2}{#3}}% P ( | ) 括弧が可変
\def\Probw@#1(#2|#3){P\fvpair[#1]{#2}{#3}}% P ( | ) 括弧が固定
\def\probW@#1(#2|#3){p\fvPair[#1]{#2}{#3}}% p ( | ) 括弧が可変
\def\probw@#1(#2|#3){p\fvpair[#1]{#2}{#3}}% p ( | ) 括弧が固定

\newcommand{\setSeparation}[3][*]{\BvPair[#1]{#2}{#3}}% \{ | \} 括弧が可変
\newcommand{\setseparation}[3][.]{\Bvpair[#1]{#2}{#3}}% \{ | \} 括弧が固定

\newcommand{\pBracket}[2][*]{\GeneralBracket{#1}(){#2}}
\newcommand{\bBracket}[2][*]{\GeneralBracket{#1}[]{#2}}
\newcommand{\BBracket}[2][*]{\GeneralBracket{#1}\{\}{#2}}
\newcommand{\vBracket}[2][*]{\GeneralBracket{#1}||{#2}}
\newcommand{\VBracket}[2][*]{\GeneralBracket{#1}\|\|{#2}}
\newcommand{\pbracket}[2][.]{\GeneralBracket{#1}(){#2}}
\newcommand{\bbracket}[2][.]{\GeneralBracket{#1}[]{#2}}
\newcommand{\Bbracket}[2][.]{\GeneralBracket{#1}\{\}{#2}}
\newcommand{\vbracket}[2][.]{\GeneralBracket{#1}||{#2}}
\newcommand{\Vbracket}[2][.]{\GeneralBracket{#1}\|\|{#2}}

\newcommand{\fnBracket}[2][*]{\@ifnextchar({\fnpBracket@{#2}{#1}}{\@ifnextchar[{\fnbBracket@{#2}{#1}}{\GenerateWarning{\string\fnBracket failed}#2}}}
\newcommand{\fnbracket}[2][.]{\@ifnextchar({\fnpbracket@{#2}{#1}}{\@ifnextchar[{\fnbbracket@{#2}{#1}}{\GenerateWarning{\string\fnbracket failed}#2}}}
\def\fnpBracket@#1#2(#3){#1\!\pBracket[#2]{#3}}% ( ) 括弧が可変(直前に関数名がくる)
\def\fnpbracket@#1#2(#3){#1\pbracket[#2]{#3}}% ( ) 括弧が可変(直前に関数名がくる)
\def\fnbBracket@#1#2[#3]{#1\!\bBracket[#2]{#3}}% [ ] 括弧が可変(直前に関数名がくる)
\def\fnbbracket@#1#2[#3]{#1\bbracket[#2]{#3}}% [ ] 括弧が可変(直前に関数名がくる)

\newcommand{\fBracket}[2][*]{\!\pBracket[#1]{#2}}% ( ) 括弧が可変
\newcommand{\fbracket}[2][.]{\pbracket[#1]{#2}}% ( ) 括弧が固定
% m は measure のイメージ:あるいはwをひっくり返したものとか:)
\def\ProbM@#1(#2){P\fBracket[#1]{#2}}% P ( ) 括弧が可変
\def\Probm@#1(#2){P\fbracket[#1]{#2}}% P ( ) 括弧が固定
\def\probM@#1(#2){p\fBracket[#1]{#2}}% p ( ) 括弧が可変
\def\probm@#1(#2){p\fbracket[#1]{#2}}% p ( ) 括弧が固定

\makeatother


\newcommand{\Norm}[2][*]{\GeneralBracket{#1}\|\|{#2}}
\newcommand{\norm}[2][.]{\GeneralBracket{#1}\|\|{#2}}

\newcommand{\trans}[1]{#1^{\bm{T}}}% not PRML-style
% 引数が既に上付き文字を伴っている場合は \trans{{A^{(2)}}} のようにする
\newcommand{\dif}[1]{\frac{\partial}{\partial #1}}
\newcommand{\diff}[2]{\frac{\partial #2}{\partial #1}}
\newcommand{\sdif}[1]{\sfrac{\partial}{\partial #1}}
\newcommand{\psdif}[2][*]{\GeneralBracket{#1}(){\sdif{#2}}}
%\newcommand{\Proba}[2]{p(#1\,|\,#2)}

\newcommand{\ddif}[2]{\frac{\partial^2}{\partial #1 \partial #2}}
\newcommand{\ddiff}[3]{\frac{\partial^2 #3}{\partial #1 \partial #2}}
\newcommand{\makeop}[1]{\mathop{\mathrm{#1}}\nolimits}
\newcommand{\sgn}{\makeop{sgn}}
\newcommand{\diag}{\makeop{diag}}
\newcommand{\tri}{\makeop{tri}}
\newcommand{\tr}{\makeop{tr}}% trace: not PRML-style
\newcommand{\KL}{\makeop{KL}}
\newcommand{\Dir}{\makeop{Dir}}
\newcommand{\Gam}{\makeop{Gam}}
\newcommand{\St}{\makeop{St}}
\newcommand{\var}{\makeop{var}}
\newcommand{\cov}{\makeop{cov}}
\newcommand{\outp}[1]{#1\trans{#1}}
\newcommand{\inp}[1]{\trans{#1}#1}
% 引数が既に上付き文字を伴っている場合は
% \outp{{x^{(1)}}}, \inp{{x^{(1)}}} のようにする
\newcommand{\vpi}{\bm{\pi}}
\newcommand{\vmu}{\bm{\mu}}
\newcommand{\vx}{\bm{x}}
\newcommand{\vvec}[2]{\begin{pmatrix} #1 \\ #2 \end{pmatrix}}
\newcommand{\hvec}[2]{\begin{pmatrix} #1 & #2 \end{pmatrix}}
\newcommand{\matt}[4]{\begin{pmatrix} #1 & #2 \\ #3 & #4 \end{pmatrix}}
\newcommand{\dett}[4]{\begin{vmatrix} #1 & #2 \\ #3 & #4 \end{vmatrix}}
\newcommand{\lmatt}[4]{\begin{pmatrix} #1 & \ldots & #2 \\ \vdots & \ddots & \vdots \\ #3 & \ldots & #4 \end{pmatrix}}
\newcommand{\quadf}[2]{\trans{\bm{#2}} #1 \bm{#2}}% not PRML-style
\newcommand{\quads}[2]{\trans{#2} #1 {#2}}
\newcommand{\CC}{\mathbb{C}}
\newcommand{\RR}{\mathbb{R}}
\newcommand{\EE}{\mathbb{E}}
\newcommand{\PhiT}{\trans{\Phi}}
\newcommand{\calN}{{\cal N}}
\newcommand{\calD}{{\cal D}}
\newcommand{\calL}{{\cal L}}
\newcommand{\calQ}{{\cal Q}}
\newcommand{\calW}{{\cal W}}
\newcommand{\calF}{{\mathcal F}}
\newcommand{\half}{\frac{1}{2}}
\newcommand{\ignore}[1]{}

\newcommand{\od}{d}% \int dx の d: not PRML-style

\newcommand{\bv}[1]{\bm{#1}}% 縦/列ベクトル: not PRML-style
\newcommand{\bM}[1]{#1}% 行列: not PRML-style
\newcommand{\oM}[1]{\bm{#1}}% 観測値データ行列: not PRML-style
\DeclareMathAlphabet{\mathssbf}{OT1}{cmss}{bx}{n}
\newcommand{\ov}[1]{\bm{#1}}% 観測値データベクトル(): not PRML-style

\newcommand{\bmm}{\bm {m}}
\newcommand{\bmLambda}{\bm{\Lambda}}
\newcommand{\bmSigma}{\bm{\Sigma}}

\newcommand{\≧}{\geqq}
\newcommand{\≦}{\leqq}
\newcommand{\⊂}{\subset}
\newcommand{\⊃}{\supset}
\newcommand{\sfrac}[2]{#1/#2}
\newcommand{\psfrac}[3][*]{\GeneralBracket{#1}(){\sfrac{#2}{#3}}}
\newcommand{\fnGamma}{\varGamma}

\RequirePackage{amsthm}
\makeatletter
\newtheoremstyle{jdefinition}
  %{\topsep}%   % ABOVESPACE
  {0pt}%   % ABOVESPACE
  %{\topsep}%   % BELOWSPACE
  {0pt}%   % BELOWSPACE
  {\normalfont}%  % BODYFONT
  {0pt}%       % INDENT (empty value is the same as 0pt)
  %{\bfseries}% % HEADFONT
  {\sf}% % HEADFONT
  {\@addpunct{.}}%         % HEADPUNCT
  {5pt plus 1pt minus 1pt}% % HEADSPACE
  {\thmname{#1}\thmnumber{\@ifnotempty{#1}{ }\@upn{#2}}%
  \thmnote{{\the\thm@notefont(#3).\hskip-.5zw\nopunct}}}%          % CUSTOM-HEAD-SPEC
\newenvironment{jproof}[1][\jproofname]{\par
  \pushQED{\qed}%
  \normalfont \topsep6\p@\@plus6\p@\relax
  \trivlist
  \item[\hskip\labelsep
        %\itshape
    #1\@addpunct{.}]\ignorespaces
}{%
  \popQED\endtrivlist\@endpefalse
}
\providecommand{\jproofname}{証明.\nopunct}
\makeatother

\iffalse% PRML-compliant style
\renewcommand{\trans}[1]{{#1}^{\!\mathrm{T}}}
\renewcommand{\od}{\mathrm{d}}% \int dx の d
\renewcommand{\bv}[1]{\mathbf{\boldsymbol{#1}}}% 縦/列ベクトル
\renewcommand{\bM}[1]{\mathbf{\boldsymbol{#1}}}% 行列
\renewcommand{\oM}[1]{\mathbf{\boldsymbol{#1}}}% 観測値データ行列
\renewcommand{\ov}[1]{\mathssbf{#1}}% 観測値データベクトル()
\renewcommand{\tr}{\makeop{Tr}}% trace
\renewcommand{\quadf}[2]{\trans{\bv{#2}} \bM{#1} \bv{#2}}
\fi

\theoremstyle{jdefinition}
\newtheorem{thm}{定理}
\newtheorem{definition}[thm]{定義}
\newtheorem{prop}[thm]{命題}
\newtheorem{cor}[thm]{系}
\newtheorem{lemma}[thm]{補題}
\newtheorem{example}[thm]{例}
\newtheorem{excercise}[thm]{問}
\newtheorem*{rem}{注意}
\newtheorem*{note}{ノート}
